% /*
%  * Filename   : correlation.tex
%  *
%  * Created    : 18.11.2016
%  *
%  * Modified   : vie 18 nov 2016 17:21:58 CET
%  *
%  * Author     : jatorre
%  *
%  * Compile    :  
%  *
%  * Purpose    :  
%  *
%  */

\documentclass{article}
\usepackage{amsmath}
\title{Correlation functions and FFT}
\author{J. A. de la Torre}
\date{\today}
\begin{document}
\maketitle

\section{The time correlation function}
We define the time correlation function of a signal $x(t)$ as
\begin{align}
  C(\tau) &= \left< x x(\tau)\right> = \frac{1}{T}\int_0^{T} dt\, x(t) x(t-\tau).
\end{align}

Let us suppose the following signal
\begin{align}
  x(t)&= \sin \left( \frac{2 \pi t}{T} \right)
\end{align}

The time correlation function is
\begin{align}
  C(\tau) &= \frac{1}{T} \int_0^{T} dt\,\sin \left( \frac{2 \pi t}{T} \right)\sin \left( \frac{2 \pi (t-\tau)}{T} \right) 
\end{align}

We use the property
\begin{align}
  \sin a \sin b &= \frac{1}{2} \left( \cos(a-b) - \cos (a+b) \right)
\end{align}
so that
\begin{align}
  C(\tau) &= \frac{1}{2} \frac{1}{T} \int_0^T dt\, \cos \left(\frac{2 \pi \tau}{T} \right)
           - \frac{1}{2} \frac{1}{T} \int_0^T dt\, \cos \left(\frac{4 \pi t}{T} - \frac{2 \pi \tau}{T} \right) \\
           &=\frac{1}{2} \frac{1}{T} \cos \left(\frac{2\pi \tau}{T}\right)  \int_{0}^{T} dt
           - \frac{1}{2} \frac{1}{T} \left. \frac{T}{4\pi} \sin \left(\frac{4\pi t}{T} - \frac{2\pi \tau}{T}\right) \right|_{0}^{T} \\
           &= \frac{1}{2} \cos \left( \frac{2 \pi \tau}{T} \right)
\end{align}

Let us compute the correlation between two signals
\begin{align}
  x(t)&= \sin \left( \frac{2 \pi t}{T} \right) \\
  y(t)&= \sin \left( \frac{4 \pi t}{T} \right)
\end{align}
so that
\begin{align}
  C(\tau) &= \left< x y(\tau) \right> \\
          &= \frac{1}{T}\int_0^{T} dt\, x(t) y(t-\tau) \\
          &= \frac{1}{2} \frac{1}{T} \int_0^T dt\, \cos\left( \frac{-2\pi t}{T} + \frac{2 \pi \tau}{T} \right)
           - \frac{1}{2} \frac{1}{T} \int_0^T dt\, \cos\left( \frac{ 6\pi t}{T} - \frac{2 \pi \tau}{T} \right) \\
           &=\frac{1}{2} \frac{1}{T} \left. \frac{-T}{2\pi} \sin \left(  \frac{-2\pi t}{T}- \frac{2 \pi \tau}{T} \right) \right|_{0}^{T}
             - \frac{1}{2} \frac{1}{T} \left. \frac{T}{6\pi} \sin \left(  \frac{6\pi t}{T}- \frac{2 \pi \tau}{T} \right) \right|_{0}^{T} \\
           &= 0
\end{align}

\section{Time correlation function using Fourier Transform}
Formally, time correlation function is defined as
\begin{align}
  C(\tau) &= \left< x x(\tau)\right> = \frac{1}{T}\int_0^{T} dt\, x(t) x(t-\tau).
\end{align}

\subsection{The continous Fourier Transform}
\subsection{The discrete Fourier Transform}
\begin{align}
  X_k &= \sum_{j=0}^{N-1} x_j e^{-2 \pi i j k / N} \\
  x_j &= \sum_{k=0}^{N-1} X_k e^{ 2 \pi i j k / N}
\end{align}
\subsection{Fast Fourier Transform}


\end{document}
