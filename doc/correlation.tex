% /*
%  * Filename   : correlation.tex
%  *
%  * Created    : 18.11.2016
%  *
%  * Modified   : mar 18 abr 2017 12:53:27 CEST
%  *
%  * Author     : jatorre
%  *
%  * Compile    :  
%  *
%  * Purpose    :  
%  *
%  */

\documentclass{article}
\usepackage{amsmath}
\usepackage[top=2cm,bottom=2cm,left=2cm,right=2cm]{geometry}
\title{Correlation functions and FFT}
\author{J. A. de la Torre}
\date{\today}
\begin{document}
\maketitle

\section{The time correlation function}
We define the time autocorrelation function of a signal $x(t)$ as
\begin{align}
  C(\tau) &= \left< x x(\tau)\right> = \frac{1}{T}\int_0^{T} dt\, x(t) x(t-\tau),
\end{align}
wher $T$ is the total length of the signal.

\subsection*{Example 1. Autocorrelation}
Let us suppose the following function
\begin{align}
  x(t)&= \sin \left( \frac{2 \pi t}{T} \right)
\end{align}

Its time autocorrelation function over a whole period $T$ is
\begin{align}
  C(\tau) &= \frac{1}{T} \int_0^{T} dt\,\sin \left( \frac{2 \pi t}{T} \right)\sin \left( \frac{2 \pi (t-\tau)}{T} \right) 
\end{align}

By using Property~\ref{eq:sinasinb} we have
\begin{align}
  C(\tau) &= \frac{1}{2} \frac{1}{T} \int_0^T dt\, \cos \left(\frac{2 \pi \tau}{T} \right)
           - \frac{1}{2} \frac{1}{T} \int_0^T dt\, \cos \left(\frac{4 \pi t}{T} - \frac{2 \pi \tau}{T} \right) \\
           &=\frac{1}{2} \frac{1}{T} \cos \left(\frac{2\pi \tau}{T}\right)  \int_{0}^{T} dt
           - \frac{1}{2} \frac{1}{T} \left. \frac{T}{4\pi} \sin \left(\frac{4\pi t}{T} - \frac{2\pi \tau}{T}\right) \right|_{0}^{T} \\
           &= \frac{1}{2} \cos \left( \frac{2 \pi \tau}{T} \right)
\end{align}

\subsection*{Example 2. Cross correlation of two functions with different periods}
Let us compute the correlation between two functions
\begin{align}
  x(t)&= \sin \left( \frac{2 \pi t}{T} \right)\,\quad &  y(t)&= \sin \left( \frac{4 \pi t}{T} \right)
\end{align}
so that the function $y(t)$ has a period $T'= \frac{1}{2}T$, being $T$ the period of $x(t)$. If we compute the time cross correlation between these two functions {\em over a period} $T$ we have
\begin{align}
  C(\tau) &= \left< x y(\tau) \right> \\
          &= \frac{1}{T}\int_0^{T} dt\, x(t) y(t-\tau) \\
          &= \frac{1}{2} \frac{1}{T} \int_0^T dt\, \cos\left( \frac{-2\pi t}{T} + \frac{2 \pi \tau}{T} \right)
           - \frac{1}{2} \frac{1}{T} \int_0^T dt\, \cos\left( \frac{ 6\pi t}{T} - \frac{2 \pi \tau}{T} \right) \\
           &=\frac{1}{2} \frac{1}{T} \left. \frac{-T}{2\pi} \sin \left(  \frac{-2\pi t}{T}- \frac{2 \pi \tau}{T} \right) \right|_{0}^{T}
             - \frac{1}{2} \frac{1}{T} \left. \frac{T}{6\pi} \sin \left(  \frac{6\pi t}{T}- \frac{2 \pi \tau}{T} \right) \right|_{0}^{T} \\
           &= 0
\end{align}

\subsection*{Example 3. Cross correlation of two functions with different periods}
Let us suppose 
\begin{align}
  x(t) &= \sin\left( \frac{2\pi t}{T}\right)\,\quad & y(t) &= \sin\left( \frac{3\pi t}{T}\right).
\end{align}
Again, that the function $y(t)$ has a period $T'= \frac{2}{3}T$, being $T$ the period of $x(t)$. If we compute the time cross correlation between these two functions {\em over a period} $T$ we have, on one hand,
\begin{align}
  \frac{1}{T} \int_0^T dt\, x(t) y(t-\tau) &=
  \frac{1}{T} \int_0^T dt\, \sin\left( \frac{2\pi t}{T}\right) \sin\left( \frac{3\pi (t-\tau)}{T}\right) \\
  &= \frac{1}{2T} \int_0^T dt\, \cos \left( \frac{-\pi t}{T} + \frac{3\pi \tau}{T}\right)
  -  \frac{1}{2T} \int_0^T dt\, \cos \left( \frac{5\pi t}{T} - \frac{3\pi \tau}{T}\right) \\
  &= 
  -\frac{1}{2\pi} \left[
    \sin\left(-\pi + \frac{3\pi\tau}{T} \right)
    - \sin\left(\frac{3\pi\tau}{T} \right)
  \right]
  -\frac{1}{10\pi} \left[
    \sin\left(5\pi - \frac{3\pi\tau}{T} \right)
    - \sin\left(\frac{-3\pi\tau}{T} \right)
  \right] \\
  &= \frac{1}{\pi} 
    \sin\left(\frac{3\pi\tau}{T} \right)
  -\frac{1}{5\pi}
    \sin\left(\frac{3\pi\tau}{T} \right)
\end{align}
where we used Properties \ref{eq:sinx-pi}, \ref{eq:sinpi-x}, and \ref{eq:sinimpar} in the last step. We finally have, then
\begin{align}
  \frac{1}{T} \int_0^T dt\, x(t) y(t-\tau) &=
 \frac{4}{5\pi} 
    \sin\left(\frac{3\pi\tau}{T} \right)
\end{align}

Note that this is {\bf different} from, on the other hand, 
\begin{align}
  \frac{1}{T} \int_0^T dt\, x(t-\tau) y(t) &=
  \frac{1}{T} \int_0^T dt\, \sin\left( \frac{2\pi (t-\tau)}{T}\right) \sin\left( \frac{3\pi t}{T}\right) \\
  &= \frac{1}{2T} \int_0^T dt\, \cos \left( \frac{-\pi t}{T} - \frac{2\pi \tau}{T}\right)
  -  \frac{1}{2T} \int_0^T dt\, \cos \left( \frac{5\pi t}{T} - \frac{2\pi \tau}{T}\right) \\
  &= 
  -\frac{1}{2\pi} \left[
    \sin\left(-\pi - \frac{2\pi\tau}{T} \right)
    - \sin\left(-\frac{2\pi\tau}{T} \right)
  \right]
  -\frac{1}{10\pi} \left[
    \sin\left(5\pi - \frac{2\pi\tau}{T} \right)
    - \sin\left(\frac{-2\pi\tau}{T} \right)
  \right] \\
  &= -\frac{1}{\pi} 
    \sin\left(\frac{2\pi\tau}{T} \right)
  -\frac{1}{5\pi}
    \sin\left(\frac{2\pi\tau}{T} \right)
\end{align}
where we again used Properties \ref{eq:sinx-pi}, \ref{eq:sinpi-x}, and \ref{eq:sinimpar} in the last step. We finally have, then
\begin{align}
  \frac{1}{T} \int_0^T dt\, x(t-\tau) y(t) &=
 -\frac{6}{5\pi} 
    \sin\left(\frac{2\pi\tau}{T} \right)
\end{align}

\subsection*{Example 4. Cross correlation of two function with the same period}
Let us suppose 
\begin{align}
  x(t) &= \sin\left( \frac{2\pi t}{T}\right)\,\quad & y(t) &= \cos \left( \frac{2\pi t}{T} \right) = \sin\left( \frac{2\pi t}{T} + \frac{\pi}{2}\right)
\end{align}

We compute, on one hand
\begin{align}
  \frac{1}{T} \int_0^T dt\, x(t) y(t-\tau) &=
  \frac{1}{T} \int_0^T dt\, \sin\left( \frac{2\pi t}{T}\right) \cos\left( \frac{2\pi (t-\tau)}{T}\right) \\
  &= \frac{1}{2T} \int_0^T dt\, \sin \left( \frac{2 \pi \tau}{T} \right)
  +  \frac{1}{2T} \int_0^T dt\, \sin \left( \frac{4\pi t}{T} - \frac{2\pi \tau}{T}\right) \\
  &= \frac{1}{2} \sin\left(\frac{2\pi\tau}{T} \right)
\end{align}
where we used Property \ref{eq:sinacosb}.

Note that this is {\bf different} from, on the other hand, 
\begin{align}
  \frac{1}{T} \int_0^T dt\, x(t-\tau) y(t) &=
  \frac{1}{T} \int_0^T dt\, \sin\left( \frac{2\pi (t-\tau)}{T}\right) \cos\left( \frac{2\pi t}{T}\right) \\
  &= \frac{1}{2T} \int_0^T dt\, \sin \left( - \frac{2\pi \tau}{T}\right)
  +  \frac{1}{2T} \int_0^T dt\, \sin \left( \frac{4\pi t}{T} - \frac{2\pi \tau}{T}\right) \\
  &= \frac{1}{2} \sin\left(\frac{-2\pi\tau}{T} \right) \\
  &= -\frac{1}{2} \sin\left(\frac{2\pi\tau}{T} \right) 
\end{align}
where we again used Properties \ref{eq:sinacosb} and \ref{eq:sinimpar} in the last step.

% \section{Time correlation function using Fourier Transform}
% Formally, time correlation function is defined as
% \begin{align}
%   C(\tau) &= \left< x x(\tau)\right> = \frac{1}{T}\int_0^{T} dt\, x(t) x(t-\tau).
% \end{align}
% 
% \subsection{The continous Fourier Transform}
% \subsection{The discrete Fourier Transform}
% \begin{align}
%   X_k &= \sum_{j=0}^{N-1} x_j e^{-2 \pi i j k / N} \\
%   x_j &= \sum_{k=0}^{N-1} X_k e^{ 2 \pi i j k / N}
% \end{align}
% \subsection{Fast Fourier Transform}


\appendix

\section*{Trigonometric relations}

\begin{align}
  \sin a \sin b &= \frac{1}{2}\left[ 
  \cos(a-b) - \cos(a+b)
\right]\label{eq:sinasinb} \\
  \sin a \cos b &= \frac{1}{2}\left[ 
  \sin(a-b) + \sin(a+b)
\right]\label{eq:sinacosb} \\
\sin(x-\pi) &= -\sin x \label{eq:sinx-pi} \\
\sin(\pi-x) &= \sin x \label{eq:sinpi-x} \\
\sin(x) &= - \sin(-x) \label{eq:sinimpar}
\end{align}

\end{document}
